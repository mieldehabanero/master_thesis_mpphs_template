\chapter{Introduction}
This chapter presents the section levels that can be used in the template.

\section{Section levels}
The following table presents an overview of the section levels that are used in
this document. The number of levels that are numbered and included in the table
of contents is set in the settings file \texttt{Settings.tex}. The levels are
shown in \cref{sec:introduction_section}\footnote{ Lorem ipsum dolor sit amet,
	consectetuer adipiscing elit. Ut purus elit, vestibulum ut, placeratac,
	adipiscing vitae, felis. Curabitur dictum gravida
	mauris.}\footnotemarksep\footnote{Lorem ipsum dolor sit amet, consectetuer
	adipiscing elit. Ut purus elit, vestibulum ut, placeratac, adipiscing vitae,
	felis. Curabitur dictum gravida mauris.}.%%%%

\begin{table}[H]
	\centering
	% NOTE: real tables should have a caption
	\begin{tabular}{ll}
		\toprule
		Name          & Command                                                           \\ \midrule
		Chapter       & \textbackslash\texttt{chapter\{\emph{Chapter name}\}}             \\
		Section       & \textbackslash\texttt{section\{\emph{Section name}\}}             \\
		Subsection    & \textbackslash\texttt{subsection\{\emph{Subsection name}\}}       \\
		Subsubsection & \textbackslash\texttt{subsubsection\{\emph{Subsubsection name}\}} \\
		Paragraph     & \textbackslash\texttt{paragraph\{\emph{Paragraph name}\}}         \\
		Subparagraph  & \textbackslash\texttt{paragraph\{\emph{Subparagraph name}\}}      \\
		\bottomrule
	\end{tabular}
\end{table}

%\footnote{\lipsum[2]}.

\section{Section} \label{sec:introduction_section}
\subsection{Subsection}
\subsubsection{Subsubsection}
\paragraph{Paragraph}
\subparagraph{Subparagraph}
\subparagraph{Subparagraph 2}
\newpage

\section{Section}
\lipsum[1]
\subsection{Subsection}
\lipsum[2]
\subsubsection{Subsubsection}
\lipsum[3]
\paragraph{Paragraph} \lipsum[4]
\subparagraph{Subparagraph}
\lipsum[5]
\subparagraph{Subparagraph 2}
\lipsum[6]